\documentclass[12pt,a4paper]{report}
\usepackage[utf8]{inputenc}
\usepackage[T1]{fontenc}
\usepackage{lmodern}
\usepackage{amsmath, amssymb}
\usepackage{graphicx}
\usepackage{hyperref}
\usepackage{booktabs}
\usepackage{geometry}
\usepackage{fancyhdr}
\usepackage{color}
\usepackage{listings}
\usepackage{float}

%----------------------------------------------------------------------
% Rapport de capitalisation pour le stage
% Ce document compile les connaissances acquises au cours des échanges
% avec des intelligences artificielles conversationnelles.  Il couvre
% les principes des caméras événementielles, les algorithmes de
% compensation de mouvement, la migration de ROS 1 vers ROS 2, la
% programmation C++ moderne (RAII, multithreading) ainsi que les
% outils et bonnes pratiques.  Les citations entre crochets
% renvoient aux références consultées lors de la recherche
% bibliographique.
%----------------------------------------------------------------------

\title{Rapport de capitalisation\bigskip\Huge Stage de synthèse en robotique}
\author{\textbf{Stagiaire :}\ Votre Nom\\\textbf{Encadrant :}\ Nom de l’encadrant\\\textbf{Période :}\ Janvier 2026}
\date{\today}

% Mise en page personnalisée
\pagestyle{fancy}
\fancyhf{}
\fancyhead[L]{Rapport de capitalisation}
\fancyhead[R]{\leftmark}
\fancyfoot[C]{\thepage}

\begin{document}

\maketitle

\tableofcontents

\chapter*{Introduction}
\addcontentsline{toc}{chapter}{Introduction}

Ce rapport de capitalisation synthétise l’ensemble des connaissances et
des échanges réalisés pendant un stage visant à développer un module
d’évitement d’obstacles dynamique pour un robot mobile équipé d’une
caméra événementielle (DAVIS 346) et fonctionnant sous ROS 2.  Les
conversations avec des intelligences artificielles ont permis
d’obtenir des explications, des conseils pratiques et des plans
d’implémentation.  Le présent document regroupe ces informations en
plusieurs chapitres pour en faire un ouvrage de référence.  Il
commence par rappeler le contexte du stage et les objectifs visés,
présente les principes fondamentaux des capteurs et des algorithmes
employés, puis détaille la migration logicielle vers ROS 2, la
programmation C++ moderne, l’intégration sur le robot Limo et les
bonnes pratiques découvertes au fil des échanges.

\chapter{Contexte du stage et objectifs}

Le stage s’inscrit dans un projet de robotique mobile où l’objectif
est de détecter et d’éviter des obstacles dynamiques en exploitant
une caméra événementielle de type DAVIS 346.  Ce capteur fournit
simultanément des flux d’événements (changement de luminosité par
pixel) et des images standard (APS), ce qui permet d’obtenir une
grande précision temporelle tout en restant compatible avec les
algorithmes de vision traditionnels.  La mission consistait à :

\begin{itemize}
  \item Comprendre l’article de l’ICRA 2023 intitulé \emph{Event‑based
  Real‑time Moving Object Detection Based on IMU Ego‑motion
  Compensation}.  Cet article propose une méthode de compensation de
  mouvement non linéaire basée sur les données d’une centrale
  inertielle (IMU) pour isoler les objets mobiles dans un flux
  d’événements.  L’approche est validée en temps réel avec une
  latence inférieure à 10 ms, ce qui est essentiel pour les drones
  ou les robots rapides.
  \item Porter le code existant de ROS 1 vers ROS 2 (Humble).  Le
  projet rpg\_dvs\_ros fournit un exemple de compensation de
  mouvement en ROS 1, mais il n’est pas complet et n’intègre pas la
  segmentation ni le regroupement d’objets.  Le passage à ROS 2
  implique de revoir la structure des nœuds, la gestion des
  paramètres, le système de build (ament\_cmake) et les politiques
  \textit{Quality of Service} (QoS).
  \item Implémenter les étapes manquantes de la chaîne de traitement
  de l’article : génération des images temporelles (\emph{time
  image}), seuillage dynamique, algorithme de clustering DBSCAN,
  et, à terme, stratégie d’évitement d’obstacles.
  \item Configurer un environnement de développement efficace
  (VS Code, colcon, compile\_commands) et assurer le bon
  fonctionnement du driver de la caméra (libcaer\_driver) ainsi que
  des messages associés (event\_camera\_msgs et dvs\_msgs).
\end{itemize}

Au fil des échanges, de nombreux concepts transverses ont été
abordés : principes des capteurs événementiels, bases et subtilités
du langage C++, conception orientée objet et RAII, multithreading,
configuration de ROS 2 (QoS, namespaces, remappings), ainsi que
les bonnes pratiques pour le développement et le débogage.

\chapter{Principes des caméras événementielles}

Les caméras événementielles représentent une rupture par rapport aux
capteurs d’images classiques.  Elles s’inspirent du fonctionnement
biologique des yeux et détectent uniquement les changements de
luminosité.  Chaque pixel travaille indépendamment et produit un
\emph{événement} lorsqu’il constate un changement d’intensité.  Ce
fonctionnement asynchrone permet d’atteindre une résolution
temporelle de l’ordre de la microseconde et une dynamique de
contraste supérieure à 120 dB, supprimant presque totalement le
flou de mouvement【86159226765082†L129-L158】.  Un capteur DAVIS
combine un détecteur d’événements (DVS) et une matrice APS capable
de fournir des images classiques.  Les flux d’événements sont donc
complétés par des images basse fréquence utiles pour la calibration
ou l’affichage【86159226765082†L198-L200】.

\section{Définition d’un événement}

Un événement est défini par un quadruplet $(t, x, y, p)$ où
\begin{itemize}
  \item $t$ est l’horodatage précis de l’événement;
  \item $(x, y)$ sont les coordonnées du pixel concerné;
  \item $p \in \{+1, -1\}$ indique la polarité (augmentation ou
  diminution de luminosité).
\end{itemize}
Contrairement à un flux vidéo à 30 images par seconde, un capteur
événementiel génère un nombre variable d’événements proportionnel
aux changements dans la scène.  Cela permet de traiter les
mouvements rapides tout en réduisant la quantité de données à
transmettre.

\section{Avantages et limites}

Les principaux avantages des caméras événementielles sont :
\begin{itemize}
  \item \textbf{Résolution temporelle élevée} : les pixels réagissent
  immédiatement aux changements, avec des latences de quelques
  microsecondes【86159226765082†L129-L158】.  Cela permet de suivre des
  objets rapides sans flou.
  \item \textbf{Plage dynamique étendue} : la dynamique de 120 dB
  dépasse de loin celle des caméras classiques, ce qui permet de
  fonctionner dans des scènes à très fort contraste【86159226765082†L129-L158】.
  \item \textbf{Réduction de la bande passante} : seuls les
  changements sont transmis, ce qui diminue fortement le nombre
  d’octets à traiter.
\end{itemize}
En contrepartie, l’interprétation des événements bruts est
complexe : les méthodes de vision traditionnelles (basées sur des
matrices d’intensité) ne s’appliquent pas directement.  Il faut
convertir les événements en représentations adaptées (images de
temps, d’accumulation, cartes de flux optique) et tenir compte du
problème d’ego‑mouvement : lorsqu’on déplace la caméra, tout le
fond génère des événements; il devient difficile de distinguer ce qui
est réellement mobile de ce qui correspond au mouvement du capteur.

\chapter{Algorithme de détection d’objets mobiles}

Cette partie résume et structure l’approche proposée dans l’article
ICRA 2023 ainsi que les compléments à implémenter.  Le pipeline
complet comprend quatre étapes : compensation de mouvement,
construction d’images temporelles, segmentation par seuillage
dynamique et clustering via DBSCAN.

\section{Compensation de mouvement non linéaire}

Le cœur de la méthode consiste à « rectifier » les événements afin
d’annuler le mouvement de la caméra.  On suppose que la rotation
dominante est mesurée par une IMU embarquée.  Soit $(x, y)$ les
coordonnées d’un événement et $t$ l’horodatage relatif au début du
paquet d’événements.  L’IMU fournit un vecteur de vitesse
angulaire $\boldsymbol{\omega} = (\omega_x, \omega_y, \omega_z)$.  Les
auteurs ont dérivé une relation non linéaire basée sur la
géométrie projective : pour chaque événement, on calcule la rotation
$\Delta\theta = \boldsymbol{\omega} \cdot t$, puis on applique le
changement de repère via les fonctions tangente et arctangente :
\begin{align*}
    X_\mathrm{n} &= \tan\bigl(\arctan((x - C_x)/f_x) - \Delta\theta_x\bigr), \\
    Y_\mathrm{n} &= \tan\bigl(\arctan((y - C_y)/f_y) - \Delta\theta_y\bigr),
\end{align*}
où $(C_x, C_y)$ sont les coordonnées du centre optique et $(f_x,
f_y)$ la focale en pixels.  Les nouveaux points $(X_\mathrm{n},
Y_\mathrm{n})$ sont ensuite convertis en indices d’image.  Cette
formule est plus précise qu’une approximation linéaire, notamment
pour des rotations rapides et pour des pixels situés en bordure du
capteur.  Les tests expérimentaux montrent un gain de précision de
10–15 \% par rapport aux approches linéaires【86159226765082†L129-L158】.

\section{Construction des images temporelles}

Après rectification, les événements de fond s’alignent.  On
construit deux représentations :

\begin{itemize}
  \item \textbf{Time Image} : pour chaque pixel, on calcule la
  moyenne des horodatages des événements qui y sont projetés.  Les
  pixels du fond ont des temps moyens plus anciens (ils reçoivent
  beaucoup d’événements tout au long du paquet), tandis que les
  objets en mouvement ont des temps récents puisqu’ils ne sont
  observés qu’à un instant donné.
  \item \textbf{Count Image} : on compte le nombre d’événements
  attribués à chaque pixel.  Cette image facilite la visualisation
  et peut servir de poids pour la segmentation.
\end{itemize}

Ces images servent d’entrée à l’étape de segmentation.

\section{Segmentation par seuillage dynamique}

Pour extraire les objets mobiles, l’article propose un seuil
dynamique $\lambda$ dépendant de la norme de la vitesse angulaire
$\lVert \boldsymbol{\omega} \rVert$ : si le temps moyen $T_{ij}$
d’un pixel $(i,j)$ est supérieur à $\lambda$, le pixel est marqué
comme appartenant à un objet dynamique.  Le seuil est adapté pour
tenir compte des variations de vitesse ; de manière générale, plus
la rotation est rapide, plus on augmente le seuil afin de ne pas
confondre le fond avec les artefacts dus à l’ego‑mouvement.  On
obtient ainsi une carte binaire séparant le fond et les régions
dynamiques.

\section{Clustering par DBSCAN}

Le nuage de pixels dynamiques résultant de la segmentation
contient du bruit.  Pour regrouper les points appartenant au même
objet, l’article utilise l’algorithme \textsc{DBSCAN}.  Ce
clustering est basé sur la densité : il identifie les points ayant
au moins $k$ voisins dans un rayon $\varepsilon$ comme des
\textit{points centraux}, relie entre eux les points densément
connectés et marque les points isolés comme du bruit【322056068821967†L328-L334】.
Le DBSCAN a l’avantage de ne pas imposer un nombre de clusters
prédéterminé et de tolérer des formes arbitraires.  Dans cette
application, les paramètres $(\varepsilon, k)$ peuvent être
ajustés en fonction de la distance estimée (par la taille apparente
des objets) et du bruit du capteur.

\section{Perspectives : suivi et évitement}

L’article se concentre sur la détection instantanée.  Pour réaliser
un système d’évitement d’obstacles, il sera nécessaire d’ajouter des
étapes de suivi dans le temps (tracking) afin d’estimer la vitesse
et la trajectoire des objets, puis de générer une commande de
contournement pour le robot.  Des algorithmes comme le filtre de
Kalman ou des trackers basés sur la corrélation d’événements pourront
être explorés.

\chapter{Migration de ROS 1 vers ROS 2}

Le code fourni dans rpg\_dvs\_ros était destiné à ROS 1.  Pour
l’intégrer au robot Limo qui utilise ROS 2 (Humble), un portage
complet est nécessaire.  Ce chapitre recense les principales
différences entre les deux versions et fournit un guide pratique.

\section{Système de build}

\begin{table}[H]
    \centering
    \caption{Comparaison des systèmes de build}
    \begin{tabular}{@{}ll@{}}
        \toprule
        Aspect                & ROS 1 (catkin)                        & ROS 2 (ament\_cmake) \\
        \midrule
        Outil de compilation & \texttt{catkin\_make} ou
        \texttt{catkin\_build} & \texttt{colcon build} \\
        Fichier principal    & \texttt{CMakeLists.txt}              & \texttt{CMakeLists.txt} avec find\_package(ament\_cmake) \\
        Gestion des dépendances & \texttt{find\_package(...)}         & \texttt{find\_package(...)} suivie de
        \texttt{ament\_target\_dependencies} \\
        Installation         & \texttt{install(TARGETS ...)} à la
        main dans ROS 1 & Installation automatique via ament pour que
        \texttt{ros2 run} trouve l’exécutable \\
        Export des commandes & optionnel (pas toujours généré) & possibilité de générer
        \texttt{compile\_commands.json} pour IntelliSense \\
        \bottomrule
    \end{tabular}
    \label{tab:build}
\end{table}

Pour porter un paquet, on remplace les instructions catkin par
\texttt{ament\_cmake}.  On doit lister explicitement toutes les
dépendances dans \texttt{find\_package} puis les lier avec
\texttt{ament\_target\_dependencies}.  Enfin, on veille à ce que
l’exécutable soit installé dans le dossier \texttt{install/lib} via
\texttt{install(TARGETS ... DESTINATION lib/\${PROJECT\_NAME})}.  Sans
cette installation, la commande \texttt{ros2 run} ne trouve pas
l’exécutable.

\section{Structure des nœuds et paramètres}

Dans ROS 1, un nœud est généralement une fonction \verb|int main| qui
instancie un \texttt{ros::NodeHandle} et crée des publishers,
subscribers et services.  Les paramètres sont stockés dans un
\emph{paramètre server} global.  En ROS 2, on favorise
l’héritage de la classe \texttt{rclcpp::Node}, ce qui permet
d’encapsuler logiquement les abonnements et publications.  Les
paramètres ne sont plus globaux : chaque nœud doit déclarer ses
paramètres dans son constructeur à l’aide de
\texttt{this->declare\_parameter(...)} puis les récupérer avec
\texttt{this->get\_parameter(...)}.  La méthode
\texttt{get\_parameter} renvoie un objet \texttt{rclcpp::Parameter}
qui doit être converti explicitement (\texttt{.as\_int()},
\texttt{.as\_double()}, etc.)【932557426256562†L286-L303】.  Les
déclarations de paramètres garantissent que le système peut valider
les types et fournir une aide via les fichiers YAML.

\section{Qualité de service (QoS)}

ROS 2 introduit la notion de politiques de communication.  Un topic
n’est connecté que si les QoS du publisher et du subscriber sont
compatibles.  Pour les capteurs haute fréquence (IMU, événements,
images), on utilise généralement le profil
\texttt{rclcpp::SensorDataQoS} qui définit une fiabilité
\emph{BEST\_EFFORT} et une profondeur adaptée.  Ce profil accepte
que certains messages soient perdus afin de ne pas bloquer la
pipeline.  Les messages de commande ou de contrôle préfèrent une
fiabilité \emph{RELIABLE} et une petite file d’attente.

\section{Namespaces et remappages}

Les noms de topics en ROS 2 sont résolus en tenant compte du
namespace du nœud.  Utiliser un nom commençant par \verb|~/| revient à
le préfixer par le nom du nœud, ce qui est pratique pour éviter les
collisions.  Les remappings se font soit dans le code (via
\texttt{remappings} dans un launch file), soit en ligne de commande.
Comprendre les règles de résolution des noms est essentiel pour
connecter correctement un driver (par exemple libcaer\_driver qui
publie sur \verb|~/events| et \verb|~/imu|) avec un nœud de
traitement.

\section{Packages ROS 2 pour les caméras événementielles}

Le groupe ros‑event‑camera maintient les paquets compatibles ROS 2 :
\begin{itemize}
  \item \textbf{libcaer\_driver} : pilote pour caméras DAVIS et
  DVXplorer; publie des messages sur \verb|~/events|,
  \verb|~/imu| et \verb|~/image_raw|.
  \item \textbf{event\_camera\_msgs} : définitions de messages
  efficaces pour les événements; ce paquet remplace avantageusement
  \texttt{dvs\_msgs} en ROS 2.  Toutefois, \texttt{dvs\_msgs} est
  toujours disponible pour assurer la compatibilité ascendante.
  \item \textbf{event\_camera\_renderer} : fournit des outils
  permettant d’afficher les événements dans RViz ou sur l’écran.
\end{itemize}
Le tableau \ref{tab:packages} résume la correspondance entre les
paquets ROS 1 et leurs équivalents ROS 2.

\begin{table}[H]
    \centering
    \caption{Migration des paquets liés aux caméras événementielles}
    \label{tab:packages}
    \begin{tabular}{@{}ll@{}}
    \toprule
    Paquet ROS 1            & Paquet ROS 2 équivalent \\ \midrule
    \texttt{davis\_ros\_driver}    & \texttt{libcaer\_driver} \\ 
    \texttt{dvs\_msgs}      & \texttt{event\_camera\_msgs} (ou \texttt{dvs\_msgs}) \\ 
    \texttt{dvs\_renderer}  & \texttt{event\_camera\_renderer} \\ 
    \bottomrule
    \end{tabular}
\end{table}

\section{Exemple de structure de nœud en ROS 2}

Le code suivant illustre la création d’un nœud de compensation en
ROS 2.  On dérive de \texttt{rclcpp::Node}, on déclare les
paramètres et on crée les abonnements avec un QoS adapté.

\begin{lstlisting}[language=C++,caption={Extrait d’un nœud ROS 2},label=lst:node]
class MotionCompensationNode : public rclcpp::Node {
public:
  MotionCompensationNode() : Node("motion_compensation") {
    // Déclaration des paramètres avec valeurs par défaut
    this->declare_parameter<int>("height", 260);
    this->declare_parameter<int>("width", 346);
    this->declare_parameter<double>("focus", 6550.0);
    // Récupération des paramètres
    this->get_parameter("height", height_);
    this->get_parameter("width",  width_);
    this->get_parameter("focus", focus_);
    // Création des abonnements avec QoS SensorData
    auto qos = rclcpp::SensorDataQoS();
    imu_sub_ = this->create_subscription<sensor_msgs::msg::Imu>(
      "~/imu", qos, std::bind(&MotionCompensationNode::imuCallback, this, std::placeholders::_1));
    events_sub_ = this->create_subscription<dvs_msgs::msg::EventArray>(
      "~/events", qos, std::bind(&MotionCompensationNode::eventsCallback, this, std::placeholders::_1));
    image_pub_ = this->create_publisher<sensor_msgs::msg::Image>("~/image_raw", 10);
  }
private:
  void imuCallback(sensor_msgs::msg::Imu::SharedPtr msg);
  void eventsCallback(dvs_msgs::msg::EventArray::SharedPtr msg);
  // ...
  rclcpp::Subscription<sensor_msgs::msg::Imu>::SharedPtr imu_sub_;
  rclcpp::Subscription<dvs_msgs::msg::EventArray>::SharedPtr events_sub_;
  rclcpp::Publisher<sensor_msgs::msg::Image>::SharedPtr image_pub_;
  int height_{};
  int width_{};
  double focus_{};
};
\end{lstlisting}

Cette structure clarifie la séparation des responsabilités : le
constructeur gère la configuration, les callbacks s’occupent du
traitement des messages, et les variables membres remplacent les
globales de la version ROS 1.

\chapter{Programmation C++ moderne}

Au cours du stage, de nombreux points de langage ont été abordés.  Ce
chapitre résume les notions essentielles pour écrire un code fiable et
performant en C++.

\section{RAII et gestion des ressources}

Le paradigme \emph{Resource Acquisition Is Initialization} (RAII)
consiste à gérer l’acquisition et la libération des ressources dans
les constructeurs et destructeurs d’objets.  Ceci garantit que les
ressources (mémoire, fichiers, mutex) sont libérées même en cas
d’exception ou de retour anticipé.  Par exemple, \texttt{std::lock\_guard}
verrouille un \texttt{std::mutex} à la construction et le libère
automatiquement à la destruction.  En utilisant RAII, on évite les
fuites et les blocages que pourrait provoquer un oubli d’appel
\texttt{unlock()}.  Il est recommandé de préférer
\texttt{std::lock\_guard} ou \texttt{std::unique\_lock} à
\texttt{mutex.lock()} / \texttt{mutex.unlock()}.

\section{Multithreading et synchronisation}

Le traitement des événements et des données IMU se fait dans des
callbacks exécutés potentiellement en parallèle.  Pour éviter les
conditions de course, on protège les buffers partagés avec des
\texttt{std::mutex}.  L’utilisation de \texttt{std::lock\_guard}
garantit la libération du verrou.  On peut également recourir à
\texttt{std::unique\_lock} lorsque l’on souhaite verrouiller et
déverrouiller à des moments précis ou utiliser un verrou
conditionnel.

Pour améliorer les performances, il est conseillé de limiter les
calculs coûteux (trigonométrie) dans les boucles et de regrouper
les copies de données.  L’utilisation d’un \texttt{rclcpp::
executors::MultiThreadedExecutor} permet à ROS 2 de lancer plusieurs
callbacks en parallèle en fonction du nombre de cœurs disponibles.

\section{std::bind et lambdas}

Lors de la création de subscribers, on transmet une fonction de
rappel.  Deux approches existent :

\begin{itemize}
  \item \textbf{\texttt{std::bind}} : lie les arguments et
  l’instance de la classe.  Par exemple :
  \texttt{std::bind(&Class::method, this, \_1)}.
  \item \textbf{Lambdas capturant \texttt{this}} : plus concis
  \texttt{[this](auto msg) \{ this->method(msg); \}}.
\end{itemize}
Les lambdas sont recommandées pour leur lisibilité et car elles
permettent d’éviter les erreurs de type.  L’essentiel est d’associer
correctement la méthode à l’instance actuelle.

\section{Gestion de la mémoire et performances}

La compensation et le clustering nécessitent de parcourir des
dizaines de milliers d’événements par seconde.  Pour maintenir le
temps réel, on veille à :

\begin{itemize}
  \item Éviter les allocations dynamiques répétées à l’intérieur des
  boucles (réserver la mémoire des vecteurs).
  \item Utiliser des types natifs (\texttt{float}, \texttt{double})
  et éviter les conversions inutiles.
  \item Exploiter l’optimisation du compilateur (options
  \texttt{-O3}) et, si nécessaire, paralléliser les boucles avec
  OpenMP ou TBB.
\end{itemize}

\chapter{Environnement de développement et outils}

\section{Configuration de VS Code}

L’IDE VS Code est pratique pour développer en ROS 2 grâce à ses
extensions.  Toutefois, l’autocomplétion (IntelliSense) doit savoir
où chercher les fichiers d’en‑tête.  Après compilation avec
\texttt{colcon build}, les headers des paquets installés se trouvent
dans \verb|install/<package>/include|.  On ajoute ce chemin dans
\texttt{.vscode/c\_cpp\_properties.json} :

\begin{lstlisting}[language=json,caption={Exemple de configuration VS Code},label=lst:cppprops]
{
    "configurations": [
        {
            "name": "Linux",
            "includePath": [
                "${workspaceFolder}/**",
                "${workspaceFolder}/install/**/include",
                "/opt/ros/humble/include/**"
            ],
            "compilerPath": "/usr/bin/gcc",
            "cStandard": "c17",
            "cppStandard": "gnu++20",
            "intelliSenseMode": "linux-gcc-x64"
        }
    ],
    "version": 4
}
\end{lstlisting}

On peut également générer un fichier \texttt{compile\_commands.json}
en passant l’option
\texttt{--cmake-args -DCMAKE\_EXPORT\_COMPILE\_COMMANDS=ON} à
\texttt{colcon build}.  Ce fichier permet à l’IDE de connaître
exactement les arguments de compilation de chaque fichier source.

\section{Utilisation de libcaer\_driver}

Pour utiliser la DAVIS 346 en ROS 2, on installe le paquet
\texttt{libcaer\_driver} via apt ou à partir des sources.  Le
driver publie plusieurs topics privés (\verb|~/events|, \verb|~/imu|,
\verb|~/image_raw|).  Il est lancé via un fichier launch qui permet de
configurer le type de caméra, l’identifiant, l’activation de
l’APS et de l’IMU, et de remapper les topics si besoin.  Voici un
extrait de lancement :

\begin{lstlisting}[language=Python,caption={Lancement du driver libcaer},label=lst:launch]
from launch import LaunchDescription
from launch.actions import DeclareLaunchArgument, OpaqueFunction
from launch.substitutions import LaunchConfiguration
from launch_ros.actions import Node

def launch_setup(context, *args, **kwargs):
    node = Node(
        package='libcaer_driver',
        executable='driver_node',
        name=LaunchConfiguration('camera_name'),
        parameters=[{
            'device_type': LaunchConfiguration('device_type'),
            'dvs_enabled': True,
            'imu_accel_enabled': True,
            'imu_gyro_enabled': True,
            'aps_enabled': False,
            'event_message_time_threshold': 1.0e-3,
        }],
        remappings=[('~/reset_timestamps', LaunchConfiguration('reset_topic'))],
    )
    return [node]

def generate_launch_description():
    return LaunchDescription([
        DeclareLaunchArgument('camera_name', default_value='event_camera'),
        DeclareLaunchArgument('device_type', default_value='davis'),
        DeclareLaunchArgument('reset_topic', default_value='~/reset_timestamps'),
        OpaqueFunction(function=launch_setup),
    ])
\end{lstlisting}

Ce fichier montre que le nom réel du nœud est déterminé par
\texttt{camera\_name}; les topics préfixés par \verb|~/| seront donc
résolus en \texttt{/event\_camera/events}, etc.  Comprendre ce
comportement est indispensable pour connecter correctement vos
abonnements.

\section{Plan de travail et recommandations}

L’organisation proposée lors des échanges prévoit un déroulement en
plusieurs phases :

\begin{enumerate}
  \item \textbf{Semaine 1–2} : portage du code ROS 1 vers ROS 2,
  création d’un paquet \texttt{datasync} avec la nouvelle
  structure, déclaration des paramètres, mise en place du QoS
  \texttt{SensorDataQoS}, compilation avec colcon, vérification des
  exécutions avec un bag de test.
  \item \textbf{Semaine 3–4} : intégration sur le robot Limo :
  installer libcaer\_driver sur la machine embarquée, calibrer
  l’alignement IMU/caméra, optimiser le code pour le processeur
  embarqué (réduction des appels trigonométriques, utilisation de
  pointeurs bruts, etc.).
  \item \textbf{Semaine 5–6} : implémentation de la segmentation
  dynamique, réglage du seuil en fonction de la vitesse, premier
  test de clustering avec DBSCAN; évaluation de la précision et du
  temps d’exécution.
  \item \textbf{Semaine 7–8} : ajout d’un suivi temporel des objets,
  amélioration de la robustesse (filtrage du bruit, fusion de
  clusters), développement d’une stratégie de navigation pour
  contourner les obstacles détectés.
\end{enumerate}

Cette planification est indicative et peut être adaptée en
fonction des contraintes de terrain (mise à disposition du robot,
fiabilité des drivers, etc.).

\chapter{Conclusion et perspectives}

Le travail de capitalisation effectué à travers ce rapport offre un
panorama complet des connaissances nécessaires pour développer une
chaîne de détection d’obstacles dynamiques avec une caméra
événementielle sous ROS 2.  Nous avons exploré les principes des
capteurs bio‑inspirés, détaillé les étapes de compensation de
mouvement non linéaire et de regroupement des événements, étudié la
migration de ROS 1 vers ROS 2 ainsi que les outils de développement
associés.  Des notions transverses de programmation C++ moderne
(RAII, multithreading, lambdas) ont été mises en avant afin de
produire un code robuste et maintenable.

La suite du projet consistera à implémenter les briques manquantes
(segmentation, clustering), à optimiser l’algorithme pour des
plateformes embarquées et à intégrer un module de navigation.
Grâce à cette base solide, le stagiaire dispose d’un véritable
manuel de référence pour affronter les problèmes rencontrés et
réutiliser les solutions décrites.

\end{document}